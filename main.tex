\documentclass{article}

% Language setting
% Replace `english' with e.g. `spanish' to change the document language
% \usepackage[english]{babel}

% Set page size and margins
% Replace `letterpaper' with `a4paper' for UK/EU standard size
\usepackage[a4paper,top=2cm,bottom=2cm,left=3cm,right=3cm,marginparwidth=1.75cm]{geometry}

% Useful packages
\usepackage{amsmath}
\usepackage{graphicx}
\usepackage[colorlinks=true, allcolors=blue]{hyperref}
\bibliographystyle{plain}


\begin{document}
\begin{titlepage}
      \begin{center}
          % University name
          \Huge
          \text{Universit\"at Innsbruck} \\
          \vspace{1cm} % Add some vertical space
          \Large
          Institut f\"ur Informatik \\
          \vspace{1cm}
          \includegraphics[width=7cm]{universitaet-innsbruck-logo-cmyk-farbe.jpg}
          
          % Title
          \Huge
          \textbf{Heuristische Optimierung der Operatorplatzierung in verteilten Stream-Verarbeitungssystemen} \\
          \vspace{3cm} % Add more vertical space
          
          % Author and Institute
          \Large
          Cedric Immanuel Sillaber \\
          Matrikel-Nr: 12211124
          
          \vspace{1cm}
          \large
          VU Einführung ins Wissenschaftliche Arbeiten \\
          \vspace{4cm}
          \normalsize
          Betreuer:\\ Prof. Dipl.-Ing. Dr. Thomas Fahringer
          
          \vfill % Fill the rest of the page with whitespace
          
          % Date
          \large
          \today
      \end{center}
  \end{titlepage}
\begin{abstract}
In den vergangenen Jahren wurden Big Data Applikationen stets populärer. 
Da die Anzahl der Daten umfangreicher wird, werden effiziente Ansätze für 
verteilte Stream-Datenverarbeitung (SVS) benötigt.
Das Problem der Operatorplatzierung ist ein entscheidender Performancefaktor. 
Für das Lösen dieses Problems gibt es jedoch keine 
effiziente Lösung. Diese Arbeit beschäftigt sich mit einer effizienten heuristischen Methode, die versucht,
die optimale Lösung zu approximieren. 
\end{abstract}

\section{Einführung}
Im Zuge der fortschreitenden Digitalisierung entwickelte sich Datenverarbeitung zu einem zentralen Aspekt der Modernität.
Der Erfolg vieler Konzerne beruht auf der Expertise, wie kontinuierliche Datenmengen effizient verarbeitet werden.
Rund um die Uhr werden Daten gesammelt, die in Echtzeit verarbeitet werden müssen. 
SVSs wie (zitat nötig) sammeln, filtern und verarbeiten Daten. Die Daten werden von einer großen Menge an 
Geräten produziert. Derartige Systeme werden beispielsweise in der Analyse von Finanzmärkten \cite{k5}, 
Verabreitung von Sozialen Netzwerk-Interaktionen und der Beobachtung von Network-Traffic \cite{k5} 
eingesetzt. Ein SVS besteht aus einer Menge von unabhängigen Operatoren, die eine spezifische
Funktionalität ausführen. Aus der Unabhängigkeit der Operatoren lässt sich die Möglichkeit folgern, 
dass die Rechner im Netzwerk (Cloud-Edge \cite{k6}) lokalisiert sind, anstatt in der Cloud.
In solch einem System stehen zahlreiche Ressourcen zur Verfügung. 
Als Operatorplatzierung bezeichnet man das Problem, 
die Operatoren im System optimal auf verfügbare Knoten zu platzieren.
Für die Evaluation solcher Modelle werden diverse Quality-of-Service Attribute herangezogen. 
Dazu gehören Durchsatz, End-zu-End Latenz und Verfügbarkeit \cite{k3} \cite{k4}.  Der Ansatz in dieser Arbeit versucht sich auf generalisierte QoS Attribute 
zu fokusieren, die einfach angepasst werden können \cite{k3}.
Diese Arbeit fokusiert sich auf einen Ansatz, der bekannte Heurisiken kombiniert und optimale Lösungen effizient approximiert. Der Ansatz bezieht sich auf keine spezifische
Implementierung und ist somit modell-frei. Somit kann diese Lösung für verschiedene Systeme angewendet werden.

Die optimale systematische Position in solch einem System stellt einen maßgeblichen Performancefaktor dar. Die Lösung dieses Problems ist jedoch NP-hard \cite{k4}.



\section{Definitionen}
%In diesem Abschnitt werden die Grundlagen verteilter Stream-Verarbeitungssysteme eingegangen. Wir geben eine formale Darstellung SVSs und geben eine Definition für das Operatorplatzierungs Problem. Zu 
SVSs basieren auf einer Menge an verteilten Computer Ressourcen, die zusammen ein komplexes System ergeben. 
Dieser Sachverhalt kann mittels Graphentheorie beschrieben werden. Grundsätzlich gibt es zwei Abstraktionen solcher Systeme. 
Erstens das Datenstrom Modell und zweitens das Ressourcen Modell. 
Beide Systeme werden mit gerichteten, gewichteten zykelfreien Graphen $G = (V,E)$ dargestellt. 

\textit{Datenstrom Modelle} werden durch $G_{svs} = (V_{svs}, E_{svs})$ beschrieben. In diesem Modell beschreiben Knoten $u \in V_{svs}$ Operatoren im System. 
Zusätzlich sind in $V_{svs}$ Datenquellen und Datenbecken enthalten. Man spricht hierbei von Datenbecken (vgl. "sink"), ein Punkt, an dem die Berechnungen der Operatoren zusammekommen. 
Zu den Operatoren gehören auch sogenannte \textit{pinned} Operatoren \cite{k3}, 
die Datenquellen und -becken beinhalten. Kanten $(u,v) \in E_{svs}$ beschreiben Datenstreams 
zwischen den Operatoren $u$ und $v$.  Ein Stream ist ein kontinuierliche Sequenz von Daten. 

\textit{Ressourcen Modelle} werden durch den Graphen $G_{res} = (V_{res}, E_{res})$ dargestellt. 
Dabei wird der logische Zusammenschluss zwischen verfügbaren Computing Ressourcen beschrieben. Der Knoten $u \in V_{res}$ 
repräsentiert solch eine Ressource. In diesem Modell beschreiben Kanten $(u,v) \in E_{res}$ 
eine logische Verknüpfung zwischen dem Rechnerressourcen $u$ und $v$.

Für jeden Operator $i \in V_{svs}$ gibt es eine Menge an Kandidatressourcen $V_{res}^i$ (Schreibweise wie in \cite{k3}).

Das Operator Problem bezeichnet eine Abbildung zwischen den genannten Modellen. Die Abbildung wird eingeschränkt,
damit die zu minimierenden QuS Attribute eingeschränkt werden. Folglich wird der optimale Kandidat $u$ in den Kandidatenressourcen $V_{res}^i$
gesucht, damit Operator $i$ auf Knoten $u$ platziert wird.
Hierbei bezieht sich das Problem auf die Inkonsistenz zwischen 
logisch benachbarten Operatoren im Ressourcen Modell $G_{res}$ und optimalen Entscheidungen der Operatoren im Datenstrom Modell $G_{svs}$.

% zusätzlich pinned operators
% vielleicht Fig. 1 aus efficient operator placement hinzufügen?.

\section{Definition von Operatorplatzierungproblem}
Um das Problem formal zu definieren, verwenden wir den binären Ausdruck $x_{i,u}$ $i \in V_{svs}, u \in V_{res}: x_{i,u} = 1$ wenn Operator $i$
auf dem Rechner $u$ platziert wird, andernfalls $x_{i,u} = 0$

\[ 
    \begin{gathered}
        \sum_{i \in V_{svs}} C_i x_{i,u} < C_u \quad \forall u \in V_{res} \\ % constrain: node resource
        \sum_{u \in V_{res}^i} x_{i,u} = 1 \quad \forall i \in V_{dsp} \\ % constrain: node only placed on candidate resources
        x_{i,u} \in \{0,1\} \quad \forall i \in V_{svs}, u \in V_{res}^i
    \end{gathered}
\] 

Mithilfe einer *penalty function* werden die  Verbindungen zwischen zwei spezifischen Knoten bezüglich der QoS Attribute bewertet. Dadruch wird ein Vergleich 
der Rechnerressourcen möglich. Dabei werden die Links zwischen $u \in V_{res}^i$ möglich. Auf die penalty function wird im folgenden eingegangen.

\section{Penalty Function}
% penalty function ist für Links, nicht für knoten!
Die Auswahl von verschiedenen Möglichkeiten $u \in V_{res}^i$ bringt Einbussen mit sich. Diverse Ressourcen haben verschiedene Lokalitäten,
deren Performance durch Netzwerkdynamiken beeinflusst wird. Da die Daten übertragen werden müssen, kommen nicht vorhersehbare Network Delays dazu. Hierzu müssen Network Delay, 
Bandbreite und Netzwerkgeschwindigkeit\cite{k3} betrachtet werden.



\subsection{Heuristiken}
Wie in \cite{k4} gezeigt, ist das Operatorplatzierungproblem NP-hard. Da die initiale Platzierung der Operatoren eine tragender Faktor
in der Performance einnimmt, werden effiziente Heuristiken benötigt. In dieser Abreit wird eine effiziente Methode vorgestellt, die 
zu einer approximierten Optimalösung führt. Dieser Ansatz beinhaltet eine Kombination mehrere bekannter Heuristiken. 
Ein Greedy First-Fit Ansatz in Kombination mit einer lokalen Suche findet meist lokale Optima. 
Um dem entgegenzuwirken wird dieser Ansatz mit einer Tabu Search verbunden. Somit werden häufiger \cite{k3} globale Optima gefunden. 

Ein Greedy First-Fit Algorithmus wird für das Bin-packing Packing Problem verwenden, 
aber auch oft für das Operator Placement Problem \cite{k7}\cite{k8}.
Da diese Heuristik meist nur lokale Optima findet, werden andere Ansätze hinzugezogen. 
Zum einen wird Local-Search verwendet, ein Verfahren, das mit einem Greedy Ansatz über einen Teil der Funktion iteriert. Da auch dieses
Verfahren dazu neigt, lokale Optima auszuwählen, wird zusätzlich Tabu Search implementiert. 
Die drei Heuristiken werden gekonnt kombiniert und führen somit zu einer besseren Approximation. 







%\bibliographystyle{alpha}
%\bibliography{sample}
\bibliography{biblio}

\end{document}